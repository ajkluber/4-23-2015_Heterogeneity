\documentclass[preprint]{elsarticle}

\usepackage{xr}
\usepackage{bm}
%\usepackage{pdfpages}
\usepackage{graphicx}
\usepackage{amsmath}


\bibliographystyle{elsarticle-num}

\begin{document}

\begin{frontmatter}

\title{Energetic heterogeneity selects the folding nucleus of PDZ protein}
%\title{Elsevier \LaTeX\ template\tnoteref{mytitlenote}}
%\tnotetext[mytitlenote]{Fully documented templates are available in the elsarticle package on \href{http://www.ctan.org/tex-archive/macros/latex/contrib/elsarticle}{CTAN}.}

%% Group authors per affiliation:
\author[6300 Main St. Houston TX, 77005]{Alexander Kluber}

\author[6300 Main St. Houston TX, 77005]{Cecilia Clementi}
%\address{6300 Main St. Houston TX, 77005}

%% or include affiliations in footnotes:
%\author[mymainaddress,mysecondaryaddress]{Elsevier Inc}
%\ead[url]{www.elsevier.com}

%\author[mysecondaryaddress]{Global Customer Service\corref{mycorrespondingauthor}}
%\cortext[mycorrespondingauthor]{Corresponding author}
%\ead{support@elsevier.com}

%\address[mymainaddress]{1600 John F Kennedy Boulevard, Philadelphia}
%\address[mysecondaryaddress]{360 Park Avenue South, New York}

\begin{abstract}
(250 word max) Energetic heterogeneity in structure-based models.
\end{abstract}

\begin{keyword}
structure-based model \sep Protein folding \sep folding nucleus
\end{keyword}

\end{frontmatter}


\section{Introduction}

    Protein folding has come to serve as an important touchstone for understanding
biomolecular organization. One where simplified theories and models have been
able to make important connections with experiment. Folding is possible on
biological timescales because evolution has crafted the interactions in the
folded structure to be in harmony when compared to the average misfolded
alternative. The

    Simplified protein models serve as an important testbed for connecting
energy landscape theory to experimental observations. In particular,
structure-based models have demonstrated that the topology of the folded state
plays an important role in shaping the energy landscape \cite{Clementi2000}.

    Since native contacts are the central component to structure-based models and

    For simplicity the strengths of the native contact interactions in
structure-based models are commonly taken to be uniform across the protein.
Since we do not know a priori the appropriate values to assign. 

    Theoretical work by Plotkin and Onuchic showed that both energetic and
structural heterogeneity can affect the folding mechanism and folding free
energy barrier. 

    Work by Cho by has shown that structure-based models of $\beta$ proteins
are more robust with respect to energetic perturbation than $\alpha$-helical
proteins. Experimental evidence allows suggests that homologous
$\alpha$-helical proteins can exhibit a wider spectrum of folding mechanisms
(Fersht Homeodomain, Clarke spectrins, Radford Im7). It is 

    By modeling energetic heterogeneity .


    Understanding the sensitivity of model predictions with respect to parameterization is
fundamental computation. For example, even amongst. We parameterize our model to be consistent with .
 choice of parameterization.
 how sensitive a model parameterization is ways.

put native contacts in .have a .    Theoretical work has shown that .
 The original 
off-lattice variation \cite{Clementi2000}.



    Despite having the potential .

    Single folding trajectories perform diffusion that is bias towards the
native state. 

    The probability fluxes of folding trajectories are guided by the contours
of the energy landscape.

    The ensemble of folding routes can be imagined as following riverbeds in
the free energy funneled landscapes. Upon close inspection these folding routes
are braided amongst the microcorrugations .

Structure-based models have now become common practice in modeling of large
biomolecules as they give access to longer timescale dynamics of biomolecules
that fold to reasonably well-defined structures. Structure-based models are 
supported theoretically by the energy landscape theory of protein folding and 
in particular the principle of minimal frustration. The principle of minimal 
frustration states that heteropolymers searching for particle folded structure.

\cite{Clementi2000}
\cite{Matysiak2004}
\cite{Matysiak2006}
\cite{Cho2009a}

Heuristically speaking .

Structure-based models started by using uniform contact strengths for all 
interactions (the homogeneous model). This would accurately describe cases
where the average contact energies is representative of 

\cite{Cho2009a}

Heterogeneous contact energies have been implemented in the literature by:
Matysiak, Clementi
Karanicolas, Brooks
Cho, Wolynes

    There are several simplified models of proteins that use funneling in their
construction. The model of Karanicolas, Brooks incorporates energetic and
backbone heterogeneity into . 

The folding mechanisms of all $\beta$ proteins are robustly captured by
structure-based models because heterogeneity in their native and non-natives
contacts is self-averaging. 

The quest for self-averaging properties.

A self-averaging properties.

Physical descriptors that characterize an esemble. For example, the radius of
gyration and average collapse time of heteropolymer is a self-averaging
property of sequences with the same hydrophobic content (average intrachain
attraction). 


Considering all the contact energy distributions with similar characteristics
(mean and variance), how do we show that ours leads to different physics?

Hypothesis: Arbitrary heterogeneity will lead to 1) broadening of the
transition and 2) lower free energy barrier.
\cite{Plotkin2002a}


    Oztop and coworkers \cite{Oztop2004} have previously suggested that real
proteins have significant energetic heterogeneity by comparing the dispersion
of $\phi$-values to dispersion in contact loop lengths:
$\overline{\delta\phi^2}$ vs. $\frac{\overline{\delta l^2}}{\overline{l^2}}$

\section{Results}

\subsection{Folding kinetics and thermodynamics}

    Energetic heterogeneity alters the folded and unfolded states: the unfolded
state has more residual structure and the folded state frays. This results in
smaller smaller energetic and entropic changes upon folding $\Delta E_{N-U}$
and $\Delta S_{N-U}$. The folding temperature usually lowers because the
difference in energy decreases more than the difference in entropy $T_f \approx
\frac{\Delta E_{N-U}}{k_B \Delta S_{N-U}}$. 

    The theory of Plotkin and Onuchic \cite{Plotkin2002a} treats energetic and
structural heterogeneity explicitly. According to the Plotkin theory a random
perturbation to the contact energies will tend to lower the free energy barrier
and this is seen in simulations of random perturbations to the homogeneous
model. 


    The configurational diffusion coefficient is not discussed in the theory of
Plotkin. Energetic heterogeneity decreases the configurational diffusion
coefficient. The theory of Portman and coworkers \cite{Portman2001a} supplies a
method for calculating the rate law prefactor in terms of the dynamics of
crossing the high-dimensional free energy saddle-point.


\subsection{Folding mechanism}

    For arbitray heterogeneity the folding nucleus shifts from globular
(located in one place) to more ramified (broken in disconnected). The overall 
locus of structure formation .
    Mechanism

 Energetic heterogeneity

 The folded state becomes frayed with increasing heterogeneity. 
    Folding temperatures usually go down because native state begins to fray

We have found that structure-based models

Adding energetic heterogeneity to the structure-based model for PDZ shifts the
center of the folding nucleus. The folding nucleus shifts from the beta1-beta2
hairpin in the homogeneous model to the C-terminus beta strands in the
heterogeneous model. 

We conclude that topology and heterogeneity are important for shaping the
folding of PDZ.


    Our procedure optimizes the model parameters in order to reproduce a given
experimental observable. We chose to optimize our model using  $\Delta\Delta
G$'s from $\phi$-value analysis (as in Matysiak 2004), however the procedure
used is general.

We claim that 

Comparisons with the Wolynes group's frustratometer suggest.

Uniform contact strengths result in fluctuations that correlate with
the growth of the folding nucleus. Contacts are collectively pulled into the
nucleus.

Heterogeneity blurs the surface of the nucleus. The folding nucleus of the .

If the prefactors are 


Heterogeneity has been addressed in theoretical models of protein folding \cite{Plotkin2002a}.

Non-Markovian memory effects alter the rate prefactor and local barrier
crossing coordinate \cite{Portman2001,Portman2001a,Plotkin1998}.

The effect of non-native interactions on folding rates \cite{Plotkin2001,Clementi2004}.

    Simulations of replicas with randomly distributed native interactions
indicate what features are shared amongst all parameter sets with the same
heterogeneity. Random heterogeneity with the same variance as our optimized
interactions represent the noise level.

    Comparing replicas of the same random native heterogeneity to our optimized
energies indicate what properties are generally shared between heterogeneous
parameter sets versus what properties are specific to our results.

    We discuss our results in the context of a Random Energy Model of folding 
that treats native heterogeneity specifically \cite{Plotkin2002a}. 
Mean-field theories of folding. 

The importance of energetic heterogeneity could be considered. 

If heterogeity is self-averaging.



\begin{figure}
\includegraphics[width=\textwidth]{figures/pdz_qearlyvsqlate_0_2.eps}
\caption{Optimizing contact energies of the PDZ domain reveal folding through
an alternative nucleus due to the coupling of energetic and entropic
fluctuations. Reaction coordinates $Q_{early}$ ($Q_{late}$) are the number of
contacts that form with more (less) than $0.5$ probabilty in homogeneous SBM
transition state. Random perturbations of the same magnitude do not exhibit the
same concerted shift in mechanism.}
\end{figure}

\section{Discussion}

    Proteins are minimally frustrated heteropolymers that can fold consistenly
to well defined structures. Since their biological function requires
interacting with other molecules, residual frustration occurs in their folded
structures. It can be important to consider.

 with respect to the function that have
evolved to perform.

    The model assumes the native state to be uniformly minimally frustrated.
This leads to a natural preference for contacts that are close in sequence,
such as helices, turns, and hairpins, due to their small entropic cost. 
However, real proteins may have localized frustration due to functional
constraints during evolution (e.g. on binding faces).  

    The interplay of the backbone bias and existence of residual frustration
can explain the three studied cases. In the cases of 

    PDZ and S6 have backbone and/or contact frustration in their folded
structures in regions that are predicted to be too structured in the transition
state in comparison with experiment. On the other hand SH3 has a minimally
frustrated turn where it compares favorably with experiment.

    How can we effectively capture local interactions that are not really
minimally frustrated using structure-based models?
    - Optimize contact interactions
    - Optimize backbone interactions
    - Optimize both contact and backbone interactions
    - Use contact strengths derived from transferable potential
    - Use transferable backbone potential

Hypothesis: If I simultaneously optimize the contact and dihedral strengths
will increased flexibility compensate for contact heterogeneity?

    Raises question, is the nonlocal contact heterogeneity/frustration also
important or the corrections only important for local interactions? 

    Sam Cho (2009) suggests that predictions will be more robust for 
proteins with a high nonlocal/local ratio. Our work indicates that the 
of .

    Flexibility modulates the impact of heterogeneity/frustration.


    However, real proteins are may have localized frustration in their folded
states from functional considerations (Ferriero), which may be alleviated when
they e.g. bind their intended partener. 

    The discrepancies of the homogeneous structure-based model with experiment
observations can be understood when considering localized frustration explains.

This is why we observe repulsive native interactions.

 have localized frustration in their native 
structures indicates that this hypothesis .  

eal proteins 

The homogeneous structure-based model assumes that protein interactions are
uniformly unfrustrated throughout. However that may not be the case given that
proteins have evolved their sequences to perform particular functions as well
as fold. The folding nucleus 1 of PDZ includes residues frustration in the turn
region.

    Future work to design the beta hairpin of PDZ to favor nucleus 1.

    Gianni 2007 has previous suggested that PDZ captures the essential features
of nucleation-condensation folding mechanisms whereby the transition state
ensemble is a diffuse nucleus center around the termini. Frustration.

    All the parameters are not free in the sense that they are correlated. The
Jacobian does not have full rank.

    Protein desing 

    See \cite{Wolynes1997} for discussion of capillarity model of folding. Issues
raised are: the scaling of the effective size of the folding nucleus; the
breadth of the interface separating folding and unfolded phases; the size
scaling of the folding barrier; applicability of mean-field versus capillarity
models of foldng; what are the size of the heterogeneity effects? ; 

    According to the capillarity model of folding \cite{Wolynes1997} energetic 
heterogeneity should roughen the interface bringing capillarity model.


    See \cite{Klimov1998b} for discussion of ``multiple folding nuclei'' (MFN) 
perspective of Thirumalai. See \cite{Chen2008}: discusses ``multiple folding
nuclei'' versus ``diffuse nucleus model'' versus BW capillarity picture.

    Shakhnovich has discussed a ``specific folding nucleus'' \cite{Shakhnovich1998}
where specific interactions must be formed.


\section{Materials and Methods}

\subsection{Structure-based model}
     
    We use a ``C$_{\alpha}$'' structure-based model derived from
\cite{Clementi2000}, where the model Hamiltonian $H = H_{bonded} +
H_{nonbonded}$ has a term that applies a local bias to the backbone
$H_{bonded}$ and a term for the long-range interactions between residue beads
$H_{nonbonded}$. The functional forms of these terms are,

\begin{align}
    H_{bonded} = \sum\limits_{bonds} &k_b (r_{ij} - r_{ij}^0)^2 + \sum\limits_{angles} k_{\theta} (\theta_{ijk} - \theta_{ijk}^0)^2 + \\
        \sum\limits_{dihedrals} &k_{\phi} [ \cos(\phi_{ijkl} - \phi_{ijkl}^0) + \frac{1}{2}\cos(3(\phi_{ijkl} - \phi_{ijkl}^0))] \\
\end{align}

\begin{equation}
    H_{non-bonded} = \sum\limits_{native} \epsilon_{ij}V^{cont}_{ij}(r_{ij}) + V^{cont ex}_{ij}(r_{ij}) \
        \sum\limits_{non-native} \epsilon^{ex}_{ij}\left(\frac{r_{ex}}{r_{ij}}\right)^{12}
\end{equation}


    Non-native contacts are given a purely repulsive potential of fixed
strength $\epsilon^{ex}_{ij} = 1$ while native contacts are allowed to be
attractive or repulsive with heterogeneous strengths. When two beads have an 
attractive interactions

Residue pairs that are not in contact in the native structure are given a purely 
repulsive 

 Contact maps were created using Shadow Map\cite{Noel2012} via the SMOG
webserver\cite{Noel2010}. 

    The starting point of our investigation is the homogeneous structure-based
model derived from \cite{Clementi200} which places one bead per residue at the
C$_{\alpha}$ positions of the backbone and creates attractive interactions 
between beads that are . 

    The energy scale of the model $\epsilon$ is constrained to its average
value of the optimization.


places one bead per residues at the alpha-carbon positions of the backbones. Beads that are in contact 
in the folded structure are given attractive (or repulsive) native contact interactions. structure-based model is based off of 



    Our simplified model places one bead at the alpha-carbon of each residue. Bead
have an excluded volume radius of 4angstroms. Residues that are in contact in
the folded structure, ``native contacts'', are given an attractive gaussian interaction. 
The strengths of the native contacts is varied by the optimization algorithm in order
to reproduce the .

A structure-based model is a simplified

\begin{align}
    H_{bonded} = \sum\limits_{bonds} &k_b (r_{ij} - r_{ij}^0)^2 + \sum\limits_{angles} k_{\theta} (\theta_{ijk} - \theta_{ijk}^0)^2 + \\
        \sum\limits_{dihedrals} &k_{\phi} [ \cos(\phi_{ijkl} - \phi_{ijkl}^0) + \frac{1}{2}\cos(3(\phi_{ijkl} - \phi_{ijkl}^0))] \\
\end{align}

    The bonded constants used in this work are taken from Clementi
et.al\cite{Clementi2000} and are (all in (kj/mol)): $k_b = 20000$, $k_{\theta}
= 40$, $k_{\phi} = 1$.

    The non-bonded Hamiltonian contains long-range interactions between beads.

\begin{equation}
    H_{non-bonded} = \sum\limits_{native} \epsilon_{ij} V^{cont}_{ij}(r_{ij}) + \
        \sum\limits_{non-native} \epsilon^{ex}_{ij}\left(\frac{r_{ex}}{r_{ij}}\right)^{12}
\end{equation}

    The Gaussian potential class takes the following forms,
\begin{equation}
    V^{G}_{ij}(r_{ij}) = - e^{\frac{-(r_{ij} - r_{ij}^0)^2}{2\sigma_{ij}}}
\end{equation}

\begin{equation}
    V^{Grep}_{ij}(r_{ij}) = \frac{1}{2}\left[ \tanh\left(-\left(\frac{r_{ij} - r_{ij}^0 -\
                                \sigma_{ij}}{\sigma_{ij}}\right)\right) + 1 \right]
\end{equation}

and requires adding an additional excluded volume to the corresponding pair in
order to ensure that the contact potential is not perturbed at its equilibrium
distance $r_{ij}^0$,
\begin{equation}
    V^{Gexc}_{ij}(r_{ij}) = \left(\frac{r_{ex}^0}{r_{ij}}\right)^{12}\left(1 - \
                    e^{\frac{-(r_{ij} - r_{ij}^0)^2}{2\sigma_{ij}}}\right)
\end{equation}

Homogeneous structure-based model developed by Clementi 
Implementation in gromacs using Shadow map, Gaussian contacts. SMOG. Noel, Lammert


%% Parameter learning
\subsection{Parameter learning}



Parameter fitting algorithm developed by Matysiak, Clementi
\begin{equation}
\label{eq:TaylorExp}
    \vec{f}^{sim}(\vec\epsilon) \;\;\approx \;\;\vec{f}^{sim}(\vec{\epsilon^{(0)}})\
                                \;\;+\;\;\bm{J}\cdot\delta\vec\epsilon 
\end{equation}

    Where (dropping the vector notation) $\delta\epsilon$ is some change in the
model parameters and $\bm{J}$ encodes how that parameter change affects our
simulation observable,

\begin{equation}
    \left(\bm{J}\right)_{ij} = \frac{\partial f^{sim}_i}{\partial\epsilon_j} 
\end{equation}

    Setting equation \ref{eq:TaylorExp} equal to the vector of experimental
observables $\overline{f}$ and collecting the error $\delta f = f^{sim} -
\overline{f}$ to the left-hand side yields,

\begin{equation}
\label{eq:dfJdeps}
    -\delta f = \bm{J}\delta\epsilon
\end{equation}

    Which can be solved for the update to the model parameters $\delta\epsilon$
that will bring the simulated observables closer to the experimental values
using standard tools for ill-posed problems (e.g. damped
least-squares\cite{Marquardt1963}; Singular Value Decomposition).\footnote{Note
that even though there may be many more model parameters than observables,
correlations between the corresponding potential energies mean that these
parameters are not all truly independent. One way to quantify the lack of
independence between parameters is by looking at the rank of the correlation
matrix $c_{ij} = \langle V_iV_j\rangle - \langle V_i\rangle \langle V_j
\rangle$, which has been found to have very low rank in this work (data
available upon request).} Overall this yields an iterative protocol for
updating the model parameters which takes the form (on iteration $n$),

\begin{equation}
    \epsilon^{(n+1)} = \epsilon^{(n)} + \delta\epsilon^{(n)}
\end{equation}

    The general procedure outlined above is iterated until satistfactory
convergence. The specific form of the Jacobian matrix $\bm{J}$ depends on the
specific observable being reproduced. Recall that the Jacobian is the partial
derivative of the observable with respect to one of the model parameters,

\begin{equation}
    \left(\bm{J}\right)_{ij} = \frac{\partial f_i}{\partial\epsilon_j} 
\end{equation}
    In most cases, the Jacobian $\bm{J}$ takes the simple form of a correlation
function. For example, if $\vec{f}$ is any mechanical observable (i.e. anything
computed solely from the coordinates), then,

\begin{equation}
    \frac{\partial f_i}{\partial\epsilon_j} = -\beta\left[ \
    \left\langle f_i \frac{\partial H}{\partial\epsilon_j}\right\rangle \
  - \langle f_i \rangle \left\langle \frac{\partial H}{\partial\epsilon_j} \right\rangle\right]  
\end{equation}

    This further simplifies for the linear Hamiltonian given by equation
\ref{eq:LinearHamiltonian},
\begin{equation} \label{eq:JacobianStructural}
    \frac{\partial f_i}{\partial\epsilon_j} = -\beta\left[ \
    \langle f_i V_j \rangle - \langle f_i \rangle \langle V_j \rangle\right]  
\end{equation}

    In fact, the higher derivatives can also be obtained because
they will always be joint cumulants of $f_i$ and the potential energies
conjugate to the model parameters, ${V_k}$ (e.g. $\frac{\partial
f_i}{\partial\epsilon_j\partial\epsilon_k}$ will be a second-order joint
cumulant between $f_i$, $V_j$, and $V_k$). If desired a higher order method
could be constructed using the second derivative in the Taylor expansion.
However since the linear method convergences rather quickly and higher order
methods are much more computationally demanding this direction hasn't been
pursued. The Jacobian used to reproduce experimental $\Delta\Delta G$'s
is slightly more involved and so is addressed in the next section.

    In order to make the simulation and experimental energy scales comparable
we put them in terms of $k_BT$ at their respective temperatures, then we
multiply the experimental $\Delta\Delta G$'s by the following ratio $r =
\frac{\overline{\Delta\Delta G_{sim}}}{\overline{\Delta\Delta G_{exp}}}$ in order
to make the averages of the two equal ($r$ is only calculated from the initial
homogeneous simulations and fixed thereafter). This can be understood as
removing the systematic error and is justified because the coarse-grain model
is defined on an arbitrary energy scale. Consequently we reproduce the true
heterogeneity in the data, which is the deviation from the mean.


    Modeling hydrophobic truncations as deletion of native contacts.

    In simulation we model a mutation by perturbing the potential energy by
$\Delta H_i$ for the $i$-th mutation,

\begin{equation}
    H'_i = H + \Delta H_i
\end{equation}

    Naturally $\Delta H_i$ is calculated by subtracting some fraction of the
contact energy from the mutated residue.

\begin{equation}
\label{eq:Perturbation}
\Delta H_i = -\sum\limits_{j} w^i_j\epsilon_{j} V_{j}
\end{equation}

where the weight $w^i_j$ is calculated as the average fraction of heavy atom
contacts lost when a mutation in the experimental structure. The purpose of the
weight $w^i_j$ is to allow for different mutations at the same site. Mutations
should perturb the energy proportional to the fraction of contacts they delete
(e.g. L30V versus L30A). Then the free energy change resulting from the
perturbation $\Delta H_i$ can be calculted using Zwanzig's relation
\cite{Zwanzig1954},

\begin{equation}
    \Delta G^X_i = -k_BT \ln\langle e^{-\beta \Delta H_i}\rangle_X
\end{equation}

Where $X \in {U,TS,N}$ indicates the state we are perturbing, the unfolded (U),
transition state (TS), or native state (N), respectively. The free energy
profile along a reaction coordinate $Q$, e.g. the number of native contacts, is
used to classify structures into states ${U,TS,N}$. The boundaries of the
states are taken as $\frac{1}{3}k_BT$ from the corresponding minimum (for U,N
states) or maximum (for TS). $Q$ has been shown to be an adequate reaction
coordinate for folding in structure-based models \cite{Cho2006}.

    Differentiating this expression with respect to our model parameter yields
the difference of perturbed and unperturbed averages,

\begin{equation}
    \frac{\partial \Delta G^X_i}{\partial\epsilon_j} = -\beta\left[ \
    \left\langle \frac{\partial H'_i}{\partial\epsilon_j} \right\rangle'_X -\
    \left\langle \frac{\partial H}{\partial\epsilon_j} \right\rangle_X \
    \right]
\end{equation}

where the primed average is taken with respect to Boltzman weights of $H_i'$,
and can also be written as,



\begin{equation}
    \frac{\partial \Delta G^X_i}{\partial\epsilon_j} = -\beta\left[ \
    \frac{\left\langle e^{-\beta \Delta H_i}\frac{\partial (H + \Delta H_i)}{\partial\epsilon_j} \
    \right\rangle_X}{\langle e^{-\beta\Delta H_i} \rangle_X} -\
    \left\langle \frac{\partial H}{\partial\epsilon_j} \right\rangle_X \
    \right]
\end{equation}

Given our linear Hamiltonian from equation \ref{eq:LinearHamiltonian} and the
perturbation in \ref{eq:Perturbation} this simplifies to,
\begin{equation}
    \frac{\partial \Delta G^X_i}{\partial\epsilon_j} = -\beta\left[ \
    \frac{\left\langle e^{-\beta\Delta H_i} (1 - w^i_j) V_j  \right\rangle_X}{\langle e^{-\beta\Delta H_i} \rangle_X} -\
    \left\langle V_j \right\rangle_X \
    \right]
\end{equation}

Finally, the Jacobian for the $\Delta\Delta G$'s is found from combining these
expressions for the individual states as follows,

\begin{align}
\label{eq:ddGs}
    \frac{\partial \Delta\Delta G^{\ddagger}_i}{\partial\epsilon_j} = \
    \frac{\partial \Delta G^{TS}_i}{\partial\epsilon_j} - \
    \frac{\partial \Delta G^{U}_i}{\partial\epsilon_j}  \\
    \frac{\partial \Delta\Delta G^{\circ}_i}{\partial\epsilon_j} = \
    \frac{\partial \Delta G^{N}_i}{\partial\epsilon_j} - \
    \frac{\partial \Delta G^{U}_i}{\partial\epsilon_j} 
\end{align}


\section*{References}

\bibliography{references}



\end{document}

